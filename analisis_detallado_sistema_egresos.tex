% !TEX program = pdflatex
\documentclass[12pt,a4paper]{article}
\usepackage[utf8]{inputenc}
\usepackage[spanish]{babel}
\usepackage[left=2.5cm,right=2.5cm,top=2.5cm,bottom=2.5cm]{geometry}
\usepackage{graphicx}
\usepackage{booktabs}
\usepackage{longtable}
\usepackage{array}
\usepackage{xcolor}
\usepackage{fancyhdr}
\usepackage{hyperref}
\usepackage{listings}
\usepackage{enumitem}
\usepackage{amsmath}
\usepackage{amssymb}
\usepackage{tikz}
\usepackage{pgfplots}
\usepackage{float}

% Configuración de colores
\definecolor{primaryblue}{RGB}{59,130,246}
\definecolor{secondarygreen}{RGB}{16,185,129}
\definecolor{warningorange}{RGB}{245,158,11}
\definecolor{dangered}{RGB}{239,68,68}
\definecolor{lightgray}{RGB}{243,244,246}

% Configuración de hyperref
\hypersetup{
    colorlinks=true,
    linkcolor=primaryblue,
    filecolor=primaryblue,
    urlcolor=primaryblue,
    citecolor=primaryblue,
    pdftitle={Análisis Detallado del Sistema de Egresos - SITEMM},
    pdfauthor={Sistema SITEMM},
    pdfsubject={Análisis Técnico del Sistema de Gestión de Egresos},
    pdfkeywords={POS, Egresos, Gestión, Análisis, Sistema}
}

% Configuración de listings
\lstset{
    basicstyle=\footnotesize\ttfamily,
    backgroundcolor=\color{lightgray},
    frame=single,
    breaklines=true,
    showstringspaces=false,
    commentstyle=\color{gray},
    keywordstyle=\color{primaryblue},
    stringstyle=\color{secondarygreen}
}

% Configuración de headers y footers
\pagestyle{fancy}
\fancyhf{}
\fancyhead[L]{\textcolor{primaryblue}{\textbf{Sistema SITEMM}}}
\fancyhead[R]{\textcolor{primaryblue}{Análisis Sistema de Egresos}}
\fancyfoot[C]{\thepage}
\renewcommand{\headrulewidth}{2pt}
\renewcommand{\headrule}{\hbox to\headwidth{\color{primaryblue}\leaders\hrule height \headrulewidth\hfill}}

% Comandos personalizados
\newcommand{\highlight}[1]{\textcolor{primaryblue}{\textbf{#1}}}
\newcommand{\success}[1]{\textcolor{secondarygreen}{\textbf{#1}}}
\newcommand{\warning}[1]{\textcolor{warningorange}{\textbf{#1}}}
\newcommand{\danger}[1]{\textcolor{dangered}{\textbf{#1}}}

\title{
    \vspace{-2cm}
    {\Huge\textcolor{primaryblue}{\textbf{ANÁLISIS DETALLADO}}} \\[0.5cm]
    {\LARGE\textcolor{secondarygreen}{\textbf{Sistema de Egresos}}} \\[0.3cm]
    {\large\textcolor{gray}{Proyecto SITEMM - Sistema POS Multi-Restaurante}} \\[0.5cm]
    \rule{\textwidth}{2pt}
}

\author{
    \textbf{Análisis Técnico Completo} \\
    \textit{Generado: \today}
}

\date{}

\begin{document}

\maketitle
\thispagestyle{fancy}

\vspace{1cm}

\begin{center}
\colorbox{lightgray}{
\begin{minipage}{0.9\textwidth}
\centering
\textcolor{primaryblue}{\Large\textbf{RESUMEN EJECUTIVO}}
\vspace{0.3cm}

El Sistema de Egresos del proyecto SITEMM es una \highlight{solución empresarial completa} para la gestión de gastos operativos en restaurantes multi-tenant. Implementa un \success{flujo de trabajo profesional} con aprobaciones, presupuestos, categorización avanzada y reportes analíticos.
\end{minipage}
}
\end{center}

\vspace{1cm}

\tableofcontents
\newpage

% =====================================================
% INTRODUCCIÓN
% =====================================================

\section{Introducción}

\subsection{Propósito del Análisis}
Este documento presenta un análisis técnico exhaustivo del \highlight{Sistema de Egresos} implementado en el proyecto SITEMM, un sistema POS multi-restaurante profesional. El análisis abarca desde la arquitectura de base de datos hasta la interfaz de usuario, pasando por la lógica de negocio y las funcionalidades avanzadas.

\subsection{Alcance del Sistema}
El Sistema de Egresos es un módulo integral que permite:
\begin{itemize}[leftmargin=*]
    \item \success{Gestión completa de gastos operativos}
    \item \success{Sistema de aprobaciones por roles}
    \item \success{Control presupuestario por categorías}
    \item \success{Flujos de trabajo automatizados}
    \item \success{Reportes y analíticas avanzadas}
    \item \success{Auditoría completa de transacciones}
\end{itemize}

% =====================================================
% ARQUITECTURA TÉCNICA
% =====================================================

\section{Arquitectura Técnica}

\subsection{Stack Tecnológico}

\subsubsection{Backend}
\begin{itemize}[leftmargin=*]
    \item \highlight{Node.js + Express.js}: Framework principal del servidor
    \item \highlight{PostgreSQL}: Base de datos relacional con triggers avanzados
    \item \highlight{JWT}: Sistema de autenticación y autorización
    \item \highlight{Winston}: Sistema de logging profesional
    \item \highlight{Express-validator}: Validación robusta de datos
\end{itemize}

\subsubsection{Frontend}
\begin{itemize}[leftmargin=*]
    \item \highlight{React 18 + TypeScript}: Framework de interfaz moderna
    \item \highlight{Tailwind CSS + Shadcn/ui}: Sistema de diseño profesional
    \item \highlight{Axios}: Cliente HTTP para comunicación con API
    \item \highlight{React Hook Form}: Gestión avanzada de formularios
    \item \highlight{Recharts}: Visualización de datos y reportes
\end{itemize}

\subsection{Patrón Arquitectónico}
El sistema implementa una \highlight{arquitectura en capas} con separación clara de responsabilidades:

\begin{enumerate}[leftmargin=*]
    \item \textbf{Capa de Presentación}: Componentes React con TypeScript
    \item \textbf{Capa de Servicios}: APIs REST con validación completa
    \item \textbf{Capa de Lógica de Negocio}: Controladores y middlewares
    \item \textbf{Capa de Datos}: Modelos y acceso a base de datos
    \item \textbf{Capa de Persistencia}: PostgreSQL con triggers automáticos
\end{enumerate}

% =====================================================
% ESTRUCTURA DE BASE DE DATOS
% =====================================================

\section{Estructura de Base de Datos}

\subsection{Diseño del Esquema}
El sistema implementa un \highlight{esquema relacional normalizado} con las siguientes tablas principales:

\subsubsection{Tabla Principal: egresos}
\begin{lstlisting}[language=SQL, caption=Estructura de la tabla egresos]
CREATE TABLE egresos (
    id_egreso SERIAL PRIMARY KEY,
    concepto VARCHAR(200) NOT NULL,
    descripcion TEXT,
    monto DECIMAL(12,2) NOT NULL CHECK (monto > 0),
    fecha_egreso DATE NOT NULL DEFAULT CURRENT_DATE,
    id_categoria_egreso INTEGER NOT NULL,
    metodo_pago VARCHAR(50) NOT NULL DEFAULT 'efectivo',
    proveedor_nombre VARCHAR(150),
    estado VARCHAR(30) DEFAULT 'pendiente',
    requiere_aprobacion BOOLEAN DEFAULT FALSE,
    es_deducible BOOLEAN DEFAULT TRUE,
    es_recurrente BOOLEAN DEFAULT FALSE,
    registrado_por INTEGER NOT NULL,
    id_sucursal INTEGER NOT NULL,
    id_restaurante INTEGER NOT NULL,
    created_at TIMESTAMP DEFAULT NOW(),
    updated_at TIMESTAMP DEFAULT NOW()
);
\end{lstlisting}

\subsubsection{Sistema de Categorías}
\begin{lstlisting}[language=SQL, caption=Tabla de categorías de egresos]
CREATE TABLE categorias_egresos (
    id_categoria_egreso SERIAL PRIMARY KEY,
    nombre VARCHAR(100) NOT NULL,
    descripcion TEXT,
    color VARCHAR(7) DEFAULT '#6B7280',
    icono VARCHAR(50) DEFAULT 'DollarSign',
    activo BOOLEAN DEFAULT TRUE,
    id_restaurante INTEGER NOT NULL
);
\end{lstlisting}

\subsubsection{Control Presupuestario}
\begin{lstlisting}[language=SQL, caption=Tabla de presupuestos por categoría]
CREATE TABLE presupuestos_egresos (
    id_presupuesto SERIAL PRIMARY KEY,
    anio INTEGER NOT NULL,
    mes INTEGER CHECK (mes BETWEEN 1 AND 12),
    id_categoria_egreso INTEGER NOT NULL,
    monto_presupuestado DECIMAL(12,2) NOT NULL,
    monto_gastado DECIMAL(12,2) DEFAULT 0,
    activo BOOLEAN DEFAULT TRUE,
    id_restaurante INTEGER NOT NULL
);
\end{lstlisting}

\subsection{Funcionalidades Avanzadas de Base de Datos}

\subsubsection{Triggers Automáticos}
El sistema implementa \highlight{triggers inteligentes} para automatización:

\begin{lstlisting}[language=SQL, caption=Trigger para actualización automática de presupuestos]
CREATE OR REPLACE FUNCTION actualizar_presupuesto_gastado()
RETURNS TRIGGER AS $$
BEGIN
    IF NEW.estado IN ('pagado', 'aprobado') THEN
        UPDATE presupuestos_egresos 
        SET monto_gastado = (
            SELECT COALESCE(SUM(monto), 0)
            FROM egresos 
            WHERE id_categoria_egreso = NEW.id_categoria_egreso
              AND estado IN ('pagado', 'aprobado')
              AND EXTRACT(YEAR FROM fecha_egreso) = presupuestos_egresos.anio
              AND EXTRACT(MONTH FROM fecha_egreso) = presupuestos_egresos.mes
        )
        WHERE id_categoria_egreso = NEW.id_categoria_egreso;
    END IF;
    RETURN NEW;
END;
$$ LANGUAGE plpgsql;
\end{lstlisting}

\subsubsection{Vistas Optimizadas}
\begin{lstlisting}[language=SQL, caption=Vista para reportes de egresos]
CREATE VIEW v_egresos_detallados AS
SELECT 
    e.id_egreso, e.concepto, e.monto, e.estado,
    ce.nombre as categoria_nombre,
    ce.color as categoria_color,
    u.username as registrado_por_username,
    s.nombre as sucursal_nombre
FROM egresos e
LEFT JOIN categorias_egresos ce ON e.id_categoria_egreso = ce.id_categoria_egreso
LEFT JOIN users u ON e.registrado_por = u.id
LEFT JOIN sucursales s ON e.id_sucursal = s.id_sucursal;
\end{lstlisting}

% =====================================================
% LÓGICA DE NEGOCIO
% =====================================================

\section{Lógica de Negocio}

\subsection{Modelos de Datos}

\subsubsection{Modelo de Egresos}
El \texttt{EgresoModel} implementa todas las operaciones CRUD con \highlight{validaciones robustas}:

\begin{itemize}[leftmargin=*]
    \item \success{getAllEgresos()}: Consulta con filtros avanzados y paginación
    \item \success{createEgreso()}: Creación con validación de categorías
    \item \success{updateEgreso()}: Actualización con control de estados
    \item \success{aprobarEgreso()}: Flujo de aprobación con auditoría
    \item \success{rechazarEgreso()}: Rechazo con comentarios obligatorios
    \item \success{marcarComoPagado()}: Cambio de estado con registro
\end{itemize}

\subsubsection{Modelo de Categorías}
El \texttt{CategoriaEgresoModel} proporciona:

\begin{itemize}[leftmargin=*]
    \item \success{Gestión completa de categorías}
    \item \success{Validación de nombres únicos}
    \item \success{Soft delete para preservar histórico}
    \item \success{Estadísticas por categoría}
    \item \success{Categorías más utilizadas}
\end{itemize}

\subsection{Controladores}

\subsubsection{Control de Acceso por Roles}
\begin{lstlisting}[language=JavaScript, caption=Validación de permisos por rol]
const validateEgresosPermissions = (requiredAction) => {
  return (req, res, next) => {
    const { rol } = req.user;
    const permissions = {
      admin: ['create', 'read', 'update', 'delete', 'approve', 'pay'],
      gerente: ['create', 'read', 'update', 'delete', 'approve', 'pay'],
      contador: ['create', 'read', 'update', 'report'],
      cajero: ['create', 'read'],
    };
    
    if (!permissions[rol]?.includes(requiredAction)) {
      return res.status(403).json({
        success: false,
        message: 'No tienes permisos para esta acción'
      });
    }
    next();
  };
};
\end{lstlisting}

\subsubsection{Validación de Montos por Rol}
\begin{lstlisting}[language=JavaScript, caption=Límites de monto por rol de usuario]
const validateMontoLimits = (req, res, next) => {
  const { monto } = req.body;
  const { rol } = req.user;
  
  const limits = {
    cajero: 500,      // 500 Bs
    contador: 2000,   // 2000 Bs
    gerente: 10000,   // 10000 Bs
    admin: Infinity,  // Sin límite
  };
  
  if (monto > limits[rol]) {
    return res.status(403).json({
      message: `Monto excede límite autorizado (${limits[rol]} Bs)`
    });
  }
  next();
};
\end{lstlisting}

% =====================================================
% INTERFAZ DE USUARIO
% =====================================================

\section{Interfaz de Usuario}

\subsection{Arquitectura del Frontend}

\subsubsection{Estructura de Componentes}
El frontend implementa una \highlight{arquitectura de componentes modular}:

\begin{itemize}[leftmargin=*]
    \item \textbf{EgresosDashboard}: Panel principal con métricas
    \item \textbf{EgresosList}: Lista paginada con filtros avanzados
    \item \textbf{EgresoModal}: Formulario completo de creación/edición
    \item \textbf{CategoriasList}: Gestión de categorías
    \item \textbf{PresupuestosList}: Control presupuestario
\end{itemize}

\subsubsection{Sistema de Estado}
\begin{lstlisting}[language=JavaScript, caption=Gestión de estado con React Hooks]
const [egresos, setEgresos] = useState<Egreso[]>([]);
const [categorias, setCategorias] = useState<CategoriaEgreso[]>([]);
const [filtros, setFiltros] = useState<FiltrosEgresos>({
  page: 1,
  limit: 20
});
const [pagination, setPagination] = useState({
  page: 1,
  pages: 1,
  total: 0
});
\end{lstlisting}

\subsection{Funcionalidades de la Interfaz}

\subsubsection{Dashboard Ejecutivo}
El dashboard proporciona una \highlight{vista ejecutiva completa}:

\begin{itemize}[leftmargin=*]
    \item \success{Métricas en tiempo real}
    \item \success{Egresos pendientes de aprobación}
    \item \success{Resumen por categorías}
    \item \success{Alertas automáticas}
    \item \success{Acciones rápidas}
\end{itemize}

\subsubsection{Formulario de Egresos}
El modal de egresos incluye:

\begin{itemize}[leftmargin=*]
    \item \success{Validación en tiempo real}
    \item \success{Información del proveedor}
    \item \success{Documentos de respaldo}
    \item \success{Opciones fiscales}
    \item \success{Configuración de recurrencia}
    \item \success{Flujo de aprobaciones visible}
\end{itemize}

\subsubsection{Sistema de Filtros}
\begin{lstlisting}[language=JavaScript, caption=Filtros avanzados implementados]
interface FiltrosEgresos {
  fecha_inicio?: string;
  fecha_fin?: string;
  id_categoria_egreso?: number;
  estado?: string;
  proveedor_nombre?: string;
  page?: number;
  limit?: number;
}
\end{lstlisting}

% =====================================================
% FUNCIONALIDADES AVANZADAS
% =====================================================

\section{Funcionalidades Avanzadas}

\subsection{Sistema de Aprobaciones}

\subsubsection{Flujo de Trabajo}
El sistema implementa un \highlight{flujo de aprobaciones robusto}:

\begin{enumerate}[leftmargin=*]
    \item \textbf{Solicitud}: Usuario crea egreso (estado: pendiente)
    \item \textbf{Revisión}: Gerente/Admin revisa la solicitud
    \item \textbf{Aprobación/Rechazo}: Decisión con comentarios
    \item \textbf{Pago}: Cajero/Contador marca como pagado
    \item \textbf{Auditoría}: Registro completo del flujo
\end{enumerate}

\subsubsection{Tabla de Flujo de Aprobaciones}
\begin{lstlisting}[language=SQL, caption=Auditoría completa de aprobaciones]
CREATE TABLE flujo_aprobaciones_egresos (
    id_flujo SERIAL PRIMARY KEY,
    id_egreso INTEGER NOT NULL,
    id_usuario INTEGER NOT NULL,
    accion VARCHAR(20) NOT NULL,
    comentario TEXT,
    fecha_accion TIMESTAMP DEFAULT NOW()
);
\end{lstlisting}

\subsection{Control Presupuestario}

\subsubsection{Alertas Automáticas}
El sistema genera \highlight{alertas inteligentes}:

\begin{itemize}[leftmargin=*]
    \item \warning{Precaución}: 70\% del presupuesto ejecutado
    \item \warning{Alerta}: 90\% del presupuesto ejecutado
    \item \danger{Excedido}: Presupuesto superado
\end{itemize}

\subsubsection{Funciones de Control}
\begin{lstlisting}[language=SQL, caption=Función para obtener presupuestos excedidos]
CREATE FUNCTION getPresupuestosExcedidos(id_restaurante INTEGER)
RETURNS TABLE(categoria_nombre VARCHAR, exceso DECIMAL) AS $$
BEGIN
    RETURN QUERY
    SELECT ce.nombre, (pe.monto_gastado - pe.monto_presupuestado)
    FROM presupuestos_egresos pe
    JOIN categorias_egresos ce ON pe.id_categoria_egreso = ce.id_categoria_egreso
    WHERE pe.id_restaurante = id_restaurante 
      AND pe.monto_gastado > pe.monto_presupuestado;
END;
$$ LANGUAGE plpgsql;
\end{lstlisting}

\subsection{Egresos Recurrentes}

\subsubsection{Automatización}
El sistema permite \highlight{egresos recurrentes automatizados}:

\begin{itemize}[leftmargin=*]
    \item \success{Configuración de frecuencia}: Diario, semanal, mensual, anual
    \item \success{Generación automática}: Función programada
    \item \success{Próxima fecha calculada}: Algoritmo inteligente
    \item \success{Historial completo}: Trazabilidad total
\end{itemize}

\begin{lstlisting}[language=SQL, caption=Función para generar egresos recurrentes]
CREATE FUNCTION generar_egresos_recurrentes()
RETURNS INTEGER AS $$
DECLARE
    egreso_rec RECORD;
    nueva_fecha DATE;
    egresos_generados INTEGER := 0;
BEGIN
    FOR egreso_rec IN 
        SELECT * FROM egresos 
        WHERE es_recurrente = TRUE 
          AND proxima_fecha_recurrencia <= CURRENT_DATE
          AND estado = 'pagado'
    LOOP
        -- Calcular próxima fecha según frecuencia
        CASE egreso_rec.frecuencia_recurrencia
            WHEN 'mensual' THEN 
                nueva_fecha := egreso_rec.proxima_fecha_recurrencia + INTERVAL '1 month';
            WHEN 'anual' THEN 
                nueva_fecha := egreso_rec.proxima_fecha_recurrencia + INTERVAL '1 year';
        END CASE;
        
        -- Crear nuevo egreso
        INSERT INTO egresos (concepto, monto, id_categoria_egreso, ...)
        VALUES (egreso_rec.concepto, egreso_rec.monto, ...);
        
        egresos_generados := egresos_generados + 1;
    END LOOP;
    
    RETURN egresos_generados;
END;
$$ LANGUAGE plpgsql;
\end{lstlisting}

% =====================================================
% REPORTES Y ANALÍTICAS
% =====================================================

\section{Reportes y Analíticas}

\subsection{Dashboards Ejecutivos}

\subsubsection{Métricas Principales}
El sistema proporciona \highlight{métricas ejecutivas en tiempo real}:

\begin{itemize}[leftmargin=*]
    \item \success{Total de egresos por período}
    \item \success{Distribución por categorías}
    \item \success{Estados de aprobación}
    \item \success{Comparativas mensuales}
    \item \success{Tendencias y proyecciones}
\end{itemize}

\subsubsection{Reportes Operativos}
\begin{itemize}[leftmargin=*]
    \item \success{Egresos pendientes por aprobar}
    \item \success{Presupuestos vs. gastos reales}
    \item \success{Proveedores más frecuentes}
    \item \success{Análisis de deducibilidad fiscal}
    \item \success{Reportes de auditoría}
\end{itemize}

\subsection{Exportación de Datos}

\subsubsection{Formatos Soportados}
\begin{itemize}[leftmargin=*]
    \item \success{Excel (XLSX)}: Reportes detallados
    \item \success{CSV}: Datos para análisis externos
    \item \success{PDF}: Reportes ejecutivos
    \item \success{JSON}: Integración con APIs
\end{itemize}

% =====================================================
% SEGURIDAD Y AUDITORÍA
% =====================================================

\section{Seguridad y Auditoría}

\subsection{Controles de Seguridad}

\subsubsection{Autenticación y Autorización}
\begin{itemize}[leftmargin=*]
    \item \success{JWT con expiración}: Tokens seguros
    \item \success{Roles granulares}: Permisos específicos
    \item \success{Middleware de validación}: Control en cada endpoint
    \item \success{Rate limiting}: Protección contra ataques
    \item \success{Validación de entrada}: Sanitización completa
\end{itemize}

\subsubsection{Auditoría Completa}
\begin{lstlisting}[language=JavaScript, caption=Middleware de auditoría]
const logEgresosActivity = (action) => {
  return (req, res, next) => {
    const originalSend = res.send;
    res.send = function(data) {
      winston.info(`[EGRESOS_AUDIT] ${action}`, {
        usuario: req.user.username,
        rol: req.user.rol,
        method: req.method,
        url: req.originalUrl,
        timestamp: new Date().toISOString(),
        ip: req.ip
      });
      originalSend.call(this, data);
    };
    next();
  };
};
\end{lstlisting}

\subsection{Integridad de Datos}

\subsubsection{Constraints de Base de Datos}
\begin{lstlisting}[language=SQL, caption=Constraints para integridad]
ALTER TABLE egresos
ADD CONSTRAINT ck_egresos_estado 
CHECK (estado IN ('pendiente', 'aprobado', 'pagado', 'cancelado', 'rechazado'));

ALTER TABLE egresos
ADD CONSTRAINT ck_egresos_monto_positivo 
CHECK (monto > 0);

ALTER TABLE egresos
ADD CONSTRAINT ck_egresos_fechas_validas 
CHECK (fecha_egreso <= CURRENT_DATE + INTERVAL '1 day');
\end{lstlisting}

% =====================================================
% RENDIMIENTO Y OPTIMIZACIÓN
% =====================================================

\section{Rendimiento y Optimización}

\subsection{Optimizaciones de Base de Datos}

\subsubsection{Índices Estratégicos}
\begin{lstlisting}[language=SQL, caption=Índices para optimización de consultas]
-- Índices para consultas frecuentes
CREATE INDEX idx_egresos_fecha_restaurante 
ON egresos(fecha_egreso, id_restaurante);

CREATE INDEX idx_egresos_categoria_fecha 
ON egresos(id_categoria_egreso, fecha_egreso);

CREATE INDEX idx_egresos_estado_restaurante 
ON egresos(estado, id_restaurante);

-- Índice parcial para egresos recurrentes
CREATE INDEX idx_egresos_recurrente 
ON egresos(es_recurrente, proxima_fecha_recurrencia) 
WHERE es_recurrente = TRUE;
\end{lstlisting}

\subsubsection{Consultas Optimizadas}
\begin{itemize}[leftmargin=*]
    \item \success{Paginación eficiente}: LIMIT/OFFSET optimizado
    \item \success{Filtros indexados}: Consultas rápidas
    \item \success{Joins optimizados}: Evitar N+1 queries
    \item \success{Agregaciones eficientes}: GROUP BY optimizado
\end{itemize}

\subsection{Optimizaciones de Frontend}

\subsubsection{Técnicas Implementadas}
\begin{itemize}[leftmargin=*]
    \item \success{Lazy loading}: Carga bajo demanda
    \item \success{Memoización}: React.memo para componentes
    \item \success{Debouncing}: En filtros de búsqueda
    \item \success{Paginación}: Carga incremental
    \item \success{Caché local}: Reducir llamadas API
\end{itemize}

% =====================================================
% EVALUACIÓN TÉCNICA
% =====================================================

\section{Evaluación Técnica}

\subsection{Fortalezas del Sistema}

\subsubsection{Arquitectura}
\begin{itemize}[leftmargin=*]
    \item[\success{✓}] \textbf{Separación de responsabilidades}: Arquitectura en capas bien definida
    \item[\success{✓}] \textbf{Escalabilidad}: Diseño preparado para crecimiento
    \item[\success{✓}] \textbf{Mantenibilidad}: Código modular y bien estructurado
    \item[\success{✓}] \textbf{Testabilidad}: Componentes aislados y testeable
\end{itemize}

\subsubsection{Funcionalidad}
\begin{itemize}[leftmargin=*]
    \item[\success{✓}] \textbf{Completitud}: Cubre todo el flujo de egresos
    \item[\success{✓}] \textbf{Flexibilidad}: Configurable por roles y permisos
    \item[\success{✓}] \textbf{Automatización}: Procesos automáticos inteligentes
    \item[\success{✓}] \textbf{Auditoría}: Trazabilidad completa
\end{itemize}

\subsubsection{Tecnología}
\begin{itemize}[leftmargin=*]
    \item[\success{✓}] \textbf{Stack moderno}: Tecnologías actualizadas
    \item[\success{✓}] \textbf{TypeScript}: Tipado fuerte y seguro
    \item[\success{✓}] \textbf{PostgreSQL}: Base de datos robusta
    \item[\success{✓}] \textbf{React 18}: Framework moderno y eficiente
\end{itemize}

\subsection{Áreas de Mejora}

\subsubsection{Funcionalidades Futuras}
\begin{itemize}[leftmargin=*]
    \item[\warning{⚠}] \textbf{Reportes avanzados}: Más visualizaciones y análisis
    \item[\warning{⚠}] \textbf{Integración contable}: Conexión con sistemas ERP
    \item[\warning{⚠}] \textbf{Notificaciones}: Sistema de alertas automáticas
    \item[\warning{⚠}] \textbf{Mobile app}: Aplicación móvil para aprobaciones
\end{itemize}

\subsubsection{Optimizaciones}
\begin{itemize}[leftmargin=*]
    \item[\warning{⚠}] \textbf{Caché distribuido}: Redis para mejor rendimiento
    \item[\warning{⚠}] \textbf{CDN}: Para archivos estáticos
    \item[\warning{⚠}] \textbf{Compresión}: Optimización de transferencia
    \item[\warning{⚠}] \textbf{Monitoreo}: Herramientas de observabilidad
\end{itemize}

% =====================================================
% CASOS DE USO
% =====================================================

\section{Casos de Uso Principales}

\subsection{Flujo Típico de Egreso}

\subsubsection{Escenario: Compra de Insumos}
\begin{enumerate}[leftmargin=*]
    \item \textbf{Cajero} registra egreso por compra de ingredientes
    \item \textbf{Sistema} valida monto contra límite del cajero (500 Bs)
    \item \textbf{Sistema} asigna estado "pendiente" automáticamente
    \item \textbf{Gerente} recibe notificación de egreso pendiente
    \item \textbf{Gerente} revisa y aprueba el egreso con comentario
    \item \textbf{Sistema} actualiza presupuesto de la categoría "Insumos"
    \item \textbf{Contador} marca el egreso como pagado
    \item \textbf{Sistema} registra el flujo completo en auditoría
\end{enumerate}

\subsection{Casos Especiales}

\subsubsection{Egreso Recurrente}
\begin{itemize}[leftmargin=*]
    \item \success{Configuración}: Alquiler mensual de \$5000
    \item \success{Automatización}: Se genera automáticamente cada mes
    \item \success{Aprobación}: Requiere aprobación de gerente
    \item \success{Seguimiento}: Historial completo de recurrencias
\end{itemize}

\subsubsection{Presupuesto Excedido}
\begin{itemize}[leftmargin=*]
    \item \warning{Alerta}: Sistema detecta exceso en categoría "Marketing"
    \item \warning{Notificación}: Gerente recibe alerta automática
    \item \warning{Decisión}: Puede aprobar excepcionalmente o rechazar
    \item \warning{Reporte}: Se incluye en reportes ejecutivos
\end{itemize}

% =====================================================
% CONCLUSIONES
% =====================================================

\section{Conclusiones}

\subsection{Evaluación General}

El Sistema de Egresos del proyecto SITEMM representa una \highlight{implementación profesional y completa} para la gestión de gastos operativos en entornos multi-restaurante. La solución demuestra:

\subsubsection{Excelencia Técnica}
\begin{itemize}[leftmargin=*]
    \item \success{Arquitectura sólida} con separación clara de responsabilidades
    \item \success{Tecnologías modernas} y stack actualizado
    \item \success{Base de datos robusta} con triggers y constraints inteligentes
    \item \success{Interfaz intuitiva} con UX profesional
\end{itemize}

\subsubsection{Funcionalidad Empresarial}
\begin{itemize}[leftmargin=*]
    \item \success{Flujos de trabajo completos} desde creación hasta pago
    \item \success{Control presupuestario} con alertas automáticas
    \item \success{Sistema de aprobaciones} por roles y montos
    \item \success{Auditoría completa} con trazabilidad total
\end{itemize}

\subsubsection{Escalabilidad y Mantenimiento}
\begin{itemize}[leftmargin=*]
    \item \success{Código modular} fácil de mantener y extender
    \item \success{Configuración flexible} adaptable a diferentes necesidades
    \item \success{Performance optimizado} con índices y consultas eficientes
    \item \success{Seguridad robusta} con validaciones y controles múltiples
\end{itemize}

\subsection{Recomendaciones}

\subsubsection{Implementación}
\begin{enumerate}[leftmargin=*]
    \item \textbf{Despliegue gradual}: Implementar por módulos para minimizar riesgo
    \item \textbf{Capacitación}: Entrenar usuarios en flujos de trabajo
    \item \textbf{Monitoreo}: Implementar herramientas de observabilidad
    \item \textbf{Backups}: Establecer rutinas de respaldo automático
\end{enumerate}

\subsubsection{Evolución Futura}
\begin{enumerate}[leftmargin=*]
    \item \textbf{Integraciones}: Conectar con sistemas contables externos
    \item \textbf{Analytics}: Implementar machine learning para predicciones
    \item \textbf{Mobile}: Desarrollar app móvil para aprobaciones
    \item \textbf{APIs}: Exponer APIs para integraciones de terceros
\end{enumerate}

\vspace{1cm}

\begin{center}
\colorbox{secondarygreen}{
\begin{minipage}{0.9\textwidth}
\centering
\textcolor{white}{\Large\textbf{VEREDICTO FINAL}}
\vspace{0.3cm}

El Sistema de Egresos de SITEMM es una \textbf{solución de nivel empresarial} que cumple con los más altos estándares de calidad técnica y funcional. Su implementación representa una \textbf{inversión estratégica} para cualquier organización que busque \textbf{profesionalizar la gestión de gastos operativos}.
\end{minipage}
}
\end{center}

\vspace{2cm}

\begin{flushright}
\textit{Análisis completado el \today}\\
\textit{Proyecto SITEMM - Sistema POS Multi-Restaurante}
\end{flushright}

\end{document}
